\documentclass{jsarticle}
\usepackage[utf8]{inputenc}

\usepackage[dvipdfmx]{graphicx}

\usepackage{times}

% 画像用のライブラリ
% \usepackage{graphicx}

\title{期末レポート1}
\author{学籍番号 : 70810025 \\ 氏名 : 神田 毬央}
\date{2023-07-15}
\begin{document}
    \maketitle

    \part{Part1}

    % フォントの変更
    {\rmfamily Roman Font} \\
    {\sffamily Sans Serif Font} \\

    {\rmfamily Roman Font です。} \\
    {\sffamily Sans Serif Font です。} \\

    % 目次を入れる
    \tableofcontents

    % 第1章の本文のフォントは Sans Serif にする
    % \sffamily % 以降はこのフォントになる

    \section{第1章}

    図\ref{fig:sample}に示す。
    表\ref{tab:sample}に示す。
    

    % 箇条書き
    \begin{itemize}
        \item 表の入れ方
        \item 表のキャプションのつけかた
        \item 表のフォーマット
        \item 画像の相互参照
        \item フォントの変更
        \item 余白の入れ方
    \end{itemize}

    \input{chapter1.tex}

    % 表を作る
    \begin{table}[htbp]
        \centering
        \caption{Sample Table}
        \label{tab:sample}
        % 職員一覧 テスト用
        \begin{tabular}
            
        \end{tabular}
    \end{table}

    % 画像を張り付ける
    \begin{figure}[htbp]
        \centering
        \includegraphics[width=10cm]{sample.png}
        \caption{Sample Image}
        \label{fig:sample}
    \end{figure}

    % 参考文献を引用
    この著書\cite{latex}を引用した。あとリガチャテストttl,Te

    % 参考文献リスト
    \begin{thebibliography}{9}
        \bibitem{latex} ほげほげ
    \end{thebibliography}


    \subsection{Next Section}
    かきくけこ

    \subsubsection{Third Section}
    さしすせそ

    % Q : paragraphってなんですか
    % A : 本文の中で、小見出しのように使う

    % 以下使用例
    \paragraph{小見出し1} 小見出しは10文字以内にすること。
    \paragraph{本文} 本文は小見出しの後に書く。100文字以内にすること。

    % \begin{comment}
    %     これはコメントです。
    % \end{comment}



\end{document}
